\index{Abstract@\emph{Abstract}}
This study presents the implementation of \texttt{hIPPYfire}, a solver for large-scale Bayesian and deterministic inverse problems governed by partial differential equations (PDEs) with infinite-dimensional parameter fields that become high-dimensional after discretization. It utilizes the same scalable algorithms introduced by its predecessor, \texttt{hIPPYlib}, such as the inexact Newton Conjugate Gradient (Newton-CG) method for the computation of the maximum \textit{aposteriori} distribution (MAP point) and the low rank-approximation of the Hessian. These algorithms exploit the fact that several PDE models of physical systems have a low-dimensional solution manifold. \texttt{hIPPYfire} computes the solution of the inverse problem at a cost independent of the parameter dimension, which is measured in terms of the number of linear forward PDE solves. However, unlike \texttt{hIPPYlib} (which is built on FEniCS), \texttt{hIPPYfire} uses Firedrake to solve the PDE governing the forward problem. Firedrake presents a unique modular structure that clearly distinguishes between the programming and mathematical aspects of the library---thereby enabling contributions from programmers and mathematicians alike and ensuring its consistent development. The functionality of the solver is validated by running it on an inverse problem that is governed by an elliptic PDE according to the Bayesian framework. The major components of the inverse problem, namely the forward problem, misfit, and prior functionals, are clearly defined and used to compute the solution of the forward problem and calculate the MAP point using the inexact Newton-CG method. The succesful computation of the forward problem solution and MAP point, coupled with the efficient abstraction in Firedrake, provide motivation to incorporate functionality into \texttt{hIPPYfire}, such as the low-rank approximations of the posterior covariance and the Hessian of the data misfit.
